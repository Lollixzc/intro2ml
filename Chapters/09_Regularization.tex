\chapter{Regularization}

\section{Introduzione}
Si noti come con il gradient descent si calcola il valore minimo della loss function sul training set.
Ci si deve pertanto preoccupare del overfitting.
Il minimo $w$ e $b$ sul training set infatti non sono tipicamente il minimo per il test set.
Questo problema viene risolto attraverso la regolarizzazione.

\section{Regolarizzatori}
I regolarizzatori sono criteri addizionali alla loss function in modo da evitare overfitting.
Prova a mantenere i parametri regolari o normali.
\`E un bias sul modello che forza che il learning preferisca certi tipi di pesi rispetto agli altri.
$$argmin_{w,b}\sum\limits_{i=1}^nloss(yy')+\lambda regularizer(w,b)$$
Tipicamente non si vogliono pesi molto grandi in quanto un piccolo cambio in una feature potrebbe causare un cambio nella predizione.
Si potrebbe anche preferire pesi di $0$ per features che non sono utili.

	\subsection{Regolarizzatori comuni}
	Nota che tutti questi regolarizzatori sono funzioni convesse, e quindi facilmente minimizzabili.
		\subsubsection{Somma dei pesi}
		Il regolarizzatore somma dei pesi penalizza di pi\`u valori piccoli e si calcola come:
		$$r(w,b)=\sum\limits_{w_j}|w_j|$$
		Si dice anche $1$-norm.
		
		\subsubsection{Somma quadratica dei pesi}
		La somma quadratica dei pesi penalizza di pi\`u valori grandi e si calcola come:
		$$r(w,b)=\sqrt{\sum\limits_{w_j}w_j^2} =||w||$$
		Si dice anche $2$-norm. 
		Tipicamente si preferisce utilizzare:
		$$\sum\limits_{w_j}w_j^2 =||w||^2$$
		Per eliminare la radice.
		
		\subsubsection{$P$-norm}
		Si intende per $p$-norm:
		$$r(w,b)=\sqrt[p]{\sum\limits_{w_j}|w_j|^p}=||w||^p$$
		Valori pi\`u piccoli di $p$ $<2$ incoraggiano vettori pi\`u sparsi, mentre valori pi\`u grandi scoraggiano pesi pi\`u grandi creando pertanto vettori con pesi pi\`u simili. 
		Attenzione che con $p$ $<1$ la funzione diventa non convessa e quindi difficile da trattare.

\section{Gradient descent e regolarizzazione}
Si nota come se scelto un modello e dimostrato che $loss+regularizer$ \`e una funzione convessa si pu\`o ancora utilizzare il gradient descent.
Si nota come per costruzione le $p$-norms sono convesse per $p\ge 1$ e pertanto:
\begin{align*}
	\dfrac{d}{dw_j}objective&=\dfrac{d}{dw_j}\sum\limits_{i=1}^n\exp(-y_i(w\cdot x_i + b))+\dfrac{\lambda}{2}||w||^2=\\
	&\cdot\\
	&\cdot\\
	&\cdot\\
	&=-\sum\limits_{i=1}^ny_ix_{ij}\exp(-y_i(w\cdot x_i+b))+\lambda w_j
\end{align*}
Pertanto con il regolarizzatore l'aggornamento \`e
$$w_j=w_j+\eta y_ix_{ij}\exp(-y_i(w\cdot x_i+b))-\eta\lambda w_j$$
Si noti pertanto come la regolarizzazione, se $w_i$ \`e positiva la riduce, mentre se \`e negativa lo aumenta muovendo $w_i$ verso lo $0$.

\section{Regolarizzazione con le $p$-norms}

	\subsection{L1}
	$$w_j = w_j + \eta(lossCorrection - \lambda sign(w_j))$$
	Popolare in quanto tende a dare soluzioni sparse, ma non \`e differenziabile e lavora unicamente per risolutori a gradient descent.
	
	\subsection{L2}
	$$w_j = w_j + \eta(lossCorrection - \lambda w_j)$$
	Popolare in quanto per qualche loss function pu\`o essere risolta direttamente.
	
	\subsection{Lp}
	$$w_j = w_j + \eta(lossCorrection - \lambda cw_j^{p-1})$$
	Meno popolare in quanto non riduce i pesi abbastanza.

	\section{Metodi di machine learning con regolarizzazione}
	\begin{multicols}{2}
		\begin{itemize}
			\item Ordinario: least squares: squared loss.
			\item Ridge regression: squared loss with L2 regularization
			\item Lasso regression: squared loss with L1 regularization
			\item Elastic regression: squared loss with L2 and L1 regularization
			\item Logistic regression: logistic loss.
		\end{itemize}
	\end{multicols}
